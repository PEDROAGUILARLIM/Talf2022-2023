
\documentclass[11pt]{article}

    \title{\textbf{Práctica 3}}
    \author{Pedro Antonio Aguilar Lima}
    \date{2022-2023}
    \usepackage{dsfont}
    \addtolength{\topmargin}{-3cm}
    \addtolength{\textheight}{3cm}
\usepackage{graphicx}
\usepackage{whilecode2}
\begin{document}

\maketitle
\thispagestyle{empty}

\section{Define the TM solution of exercise 3.4 of the problem list and test its correct behaviour. 
\\ 
\\
3.4. Prove that the function $add(x,y)=x+y, with \ x,y \in  \mathds{N} $ is Turing-computable using the unary notation $ \{ | \} $ . You have to create a TM with two arguments
separated by a blank symbol that stars and ends behind the stings.}
\includegraphics[scale=0.50]{/home/kali/Escritorio/maquinaturing3.4.png}


\newpage

\section{Define a recursive function for the sum of three values.}
 
Se usa la funcion definida de adicción: el programa haría algo parecido a lo  siguiente, es decir primero sumará los primeros dos elementos y luego al resultado que esto ofrezca, le suma el siguiente término:
\\

 $suma3=<<\pi^3_1|\sigma(\pi^3_2)>|\sigma(\pi^3_3)>$
 \\
 
 \includegraphics[scale=0.60]{/home/kali/Escritorio/5.png}
 
\newpage

\section{Implement a WHILE program that computes the sum of thre values. You must use an auxiliary variable that accumulates the result of the sum.}

\begin{whilecode}[H]
$X_4 \Assig 0$

 \While{$X_1 \not = 0$}{

  $X_4 \Assig X_4 + 1$\;
  $X_1 \Assig X_1 - 1$\;
  

 }
 
  \While{$X_2 \not = 0$}{

  $X_2 \Assig X_2 - 1$\;
  $X_4 \Assig X_4 + 1$\;

 }
 
  \While{$X_3 \not = 0$}{
  
  $X_4 \Assig X_4 + 1$\;
  $X_3 \Assig X_3 - 1$\;
  

 }

 \end{whilecode}

\end{document}

